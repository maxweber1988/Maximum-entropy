\subsection{The Maximum Entropy Method (MEM)}

Given a N dimensional noisy data set $\vec{G}$ and a model characterized by the M dimensional parameter vector $\vec{A}$. 
The model is assumed to represent a valid relation between $G_{i}$ and $\vec{A}$. This means that for an exact value $G_{i}$ 
and perhaps some other known parameters, one could in principle determine the correct vector of unknown parameters $\vec{A}$. 
However since $\vec{G}$ is noisy the uncertainty 
about the true value of $G_{i}$ induces a uncertainty on the true value of $\vec{A}$. Therefore it makes sense to determine 
a probability distribution $p(\vec{A}|\vec{G})$ instead of one single solution for $\vec{A}$. \\
Using Bayes' theorem the following relation for the wanted prob. distribution $p(\vec{A}|\vec{G})$ can be found

\begin{displaymath}
 p(\vec{A}|\vec{G}) \propto p(\vec{G}|\vec{A}) p(\vec{A})
\end{displaymath}

\noindent This is the well known $posterior \propto likelihood * prior$ relation from Bayesian statistics. Hence if it 
is possible to find the likelihood and posterior distributions one has quantitative information about the probability for 
$\vec{A}$ to be true given $\vec{G}$.\\

\subsubsection{The likelihood function}

Following Bryan [reference] for the maximum entropy method the likelihood function is restricted to the functional form 
\begin{displaymath}
 p(\vec{G}|\vec{A}) \propto exp(-L(\vec{F},\vec{G}))
\end{displaymath}
where $\vec{F}$ is leneary related to $\vec{A}$, $\vec{F} = \textbf{K} \vec{A}$ by the matrix \textbf{K} which 
represents a valid relation between $\vec{A}$ and $\vec{G}$.\\

\subsubsection{The prior distribution}

In contrast to the likelihood function which can be found by data manipulation and reasoning in most cases very exact, the 
prior is quiet hard to find in a correct way. This is the typical drawback of Bayesian inference methods.
The core of the maximum entropy method is now the used prior distribution. Following again Brayn [reference] the prior for 
an unnormalized positive additive density $\vec{A}$ is given by
\begin{displaymath}
 p(\vec{A}| \alpha, \vec{m}) \propto exp(\alpha S(\vec{m},\vec{A}))
\end{displaymath}
where $\alpha \in \mathbb{R}^+$ is a unknown parameter and S is the entropy of $\vec{A}$ relative to a default model $\vec{m}$
\begin{equation}
 S = \sum_{m=1}^{M} A_m -m_m -A_m log(A_m/m_m)
 \label{eq:S(A,m)}
\end{equation}

\noindent It is noteworthy that on the one hand this restricts the parameter vector $\vec{A}$ to be possible to be interpreted as 
\textit{positive additive density}. On the other hand the choice for $\alpha$ and $\vec{m}$ are of big influece and have to 
handled with care. This subject will be addressed in more detail in a later chapter. 

\subsubsection{The posterior distribution}

\noindent Combining now both general forms of the likelihood and prior distribution within the maximum entropy framework the posterior 
distribution for $\vec{A}$ is up to a normalization constant given by 
\begin{equation}\label{eq:posterior_general}
 p(\vec{A}|\vec{m},\alpha,\vec{G}) \propto exp(\alpha S(\vec{m},\vec{A}) - L(\vec{F},\vec{G})) = exp(Q)
\end{equation}


\noindent Hence for given $\alpha$, expression~(\ref{eq:posterior_general}) reaches its maximum probability for $\vec{\hat{A}}$
which maximizes $Q = \alpha S - L$. This means we have to find $\vec{\hat{A}}$ such that
\begin{equation}\label{eq:max_A_constraint}
 \vec{\nabla} Q(\vec{\hat{A}}) = \alpha \vec{\nabla} S(\vec{\hat{A}}) - \vec{\nabla} L(\vec{\hat{A}}) = 0
\end{equation}

\noindent The numercial solution of equation~(\ref{eq:max_A_constraint}) is the central issue of the MEM algorithm. 

\subsection{Analytic Continuation of QMC Data using MEM}

Most quantum Monte Carlo (QMC) simulations produce Green's functions $G(\tau)$ of Matsubara imaginary time $\tau = it$.
However the real time/frequency results $G(t)$/$G(\omega)$ are crucial since most experiments probe quantities related to 
the real time/frequency Green's functions. Fortunately the relation between $G(\tau)$ and the imaginary part of $G(\omega)$,
is linear and given by 
\begin{equation}\label{eq:connection_real_frequency_imag_time}
G(\tau) = \int d\omega A(\omega)K(\tau,\omega) 
\end{equation}
\noindent where the so caled Lehmann spectral function is given by $A(\omega) = -\frac{1}{\pi} \Im G(\omega)$ and $K(\tau,\omega)$ is a kernel, 
different for fermionic, bosonic or anomalous case. Therefore if it is possible to reconstruct $A(\omega)$ from given $G(\tau)$
one has the information about the real frequency Green's function $G(\omega)$. Why???? \\
In this report we will restrict our self to the fermoinic case. For fermions the Lehmann spectral function is positve 
definite, $G(\tau)$ is periodic with inverse temperatur $\beta = 1 / k_B T$ and the Kernel is given by
\begin{equation}\label{eq:ferminoc_kernel}
 K(\tau,\omega) =  \frac{exp(-\tau \omega)}{1 + exp(-\beta \omega)}
\end{equation}

\subsubsection{Discretized version of the problem}

\noindent Because of the methodically given uncertainty of Quantum Monte Carlo simulations, doing QMC for N different imaginary 
times $\tau_n$ will produce a N dimensional noisy data set $\vec{G}$ where $G_n$ is the mean of all QMC steps.\\
The idea is now to find a way to extract $A(\omega)$ form the noisy data set $\vec{G}$ using the maximum entropy method.\\
First of all we note that for the MEM formalism a valid model wich predicts $G_n$ for a given M dimensional parameter vector
$\vec{A}$ is necessary. This can be achieved if expression~(\ref{eq:connection_real_frequency_imag_time}) is approximated
as Rieman sum 
\begin{equation}\label{eq:kernel_as_reiman_sum}
 G_n = G(\tau_n) = \int_{a}^{b} d \omega A(\omega)K(\tau_n,\omega) \approx \sum_{m=1}^M  A_m K(\tau_n,\omega_m)
\end{equation}
\noindent where $A_m = \Delta \omega A(\omega_m)$, $ \omega_m = \Delta \omega m$ and $\Delta \omega = (b-a)/M$ (a,b have to be choosen in a sensible way).
After this discretizaton we have a parameter vector $\vec{A} =(A_1,...,A_M)^T$ and a true linear model $\vec{G} = \textbf{K} 
\vec{A}$ where $\textbf{K} \in \mathbb{R}^{N\times M}$ and $K_{nm} = K(\tau_n,\omega_m)$.

\subsubsection{The likelihood function}

As already mentioned to apply the Maximum Entropy Method the likelihood function hast to have the functional form 
$ p(\vec{G}|\vec{A}) \propto exp(-L(\vec{F},\vec{G}))$ where $\vec{F}$ is leneary related to $\vec{A}$ by
$\vec{F} = \textbf{K} \vec{A}$ what is already fulfilled by~(\ref{eq:kernel_as_reiman_sum}). For QMC data it is 
possible to achieve a multivariate gaussian shape of the likelihood function, such that .
\begin{equation}
	L = 1/2 ((\vec{G} - \vec{F})^T diag\{1/ \sigma_n^2 \} (\vec{G} - \vec{F}))
	\label{eq:L(G)}
\end{equation}
Since for the purpose of this educational project work no real QMC data was available we will only give a short overview of
the main aspects how QMC data has to be manipulated in principle to reach the desired form of the likelihood function.\\
For each of the N imaginary times $\tau_n$, $G(\tau_n) = G_n$ is calculated a plenty of times in $N_{QMC}$ QMC steps, with results $G_n^i$ and each with a different error. 
Hence one can interpret the relative freqency as probability distribution \[ p_{QMC}(G_n)= n(G_n^i = G_n)/N_{QMC} \]
The resulting distribution $p_{QMC}(G_n)$ is not gaussian and also correlated between different QMC steps. To get rid
of this problem one perfoms a rebinning of the data. This means one considers the average of $n_b$ succeding measurement as new 
datapoint \[G_n^b = \sum_{(b-1) n_b + 1}^{b n_b} \frac{G_n^i}{n_b} \] 
\noindent So instead of $N_{QMC}$ datapoints for each $\tau_n$
we have now $N_b = N_{QMC}/n_b$ datapoints. This rebinning has now two desired effects wich we can understand if the
procedure is considered as logical 2 step rebinning. 
\begin{enumerate}
\item In a first rebinning we get rid of the correlations between the succeding QMC steps. (As long as the bin size $n_b$
 is choosen big enough compared to the correlation lenght.)
\item Since correlations are removed the rebinned data represents a set of independent and identical drawn random variables.
Hence for the second rebinning step we can argue using the central limit theorem that the resulting random variable should
be gaussian distributed.
\end{enumerate}

Remark: To find the optimal binsize $n_b$ the current method is to compare higher moments (skewness, kurtosis) of the rebinned 
data to a data set of equal length $N_b$, drawn by perfect gaussian. The optimal size for $n_b$ is reached if the
moments for both data samples converge.\\

\noindent This gaussian distribution can be approximated by \[ p(G_n^b) = \frac{1}{\sqrt{2 \pi} \sigma_n} exp(-\frac{(G_n^b - \overline{G}_n)^2}{2 \sigma_n^2}) \]
\noindent where $\overline{G}_n = \sum_{b} G_n^b /N_b$ and $\sigma_n^2 = \sum_b (G_n^b - \overline{G}_n)^2 / (N_b -1)$ are calculated
in the usual way from the data. It is helpfull to keep in mind, that this distributions is only an approximation and not the 
true one due to the errors in $\overline{G}_n$ and $\sigma_n$.\\
But with this information we can argue using again the CLT that the true distribution of the mean for each time step $\tau_n$ is 
again gaussian 
\[ p(\overline{G}_n) = \frac{1}{\sqrt{2 \pi} \sigma_n^{real}} exp(-\frac{(\overline{G}_n - \mu_n)^2}{2 \sigma_n^{real}}) \]

where $\mu_n$ and $\sigma_n^{real}$ represent unknown true values. We now assume that for a true $G(\tau_n)$ the observed QMC 
results are distributed in a way that $G(\tau_n)$ is given by the mean of the observed QMC results. Since the mean is conserved 
under all rebinning and averaging steps we have $G(\tau_n) = \mu_n$. Using ~(\ref{eq:kernel_as_reiman_sum}) we can argue
that for a given spectral function $A(\omega)$ the observed distribution for $\overline{G}_n$ is given by 

\[ p(\overline{G}_n | \vec{A}) = \frac{1}{\sqrt{2 \pi} \sigma_n^{real}} exp(-\frac{(\overline{G}_n - \sum_m K_{nm}A_m)^2}{2 \sigma_n^{real}}) \]

\noindent In a last step we approximate $\sigma_n^{real} \approx \sigma_n/ \sqrt N_b$ motivated by the CLT and reformulate the 
whole problem in a compact multivariate form. This gives a likelihood function 

\begin{equation}
 p(\vec{\overline{G}}|\vec{A}) \propto exp \left( -\frac{1}{2} (\vec{\overline{G}}-\textbf{K}\vec{A})^T diag\{ \frac{N_b}{\sigma_n^2} \} (\vec{\overline{G}}-\textbf{K}\vec{A})  \right)
\end{equation}

\noindent where we have assumed statistical independence between the different $\overline{G}_n$. For simplicity of notation 
we will set $\vec{\overline{G}} \rightarrow \vec{G}$ and $  \sigma_n / \sqrt N_b  \rightarrow \sigma_n$ in the upcomming parts
of this report. Such that the final form of the likelihood function given by 

\begin{align}\label{eq:likelihood function}
 \begin{split}
 p(\vec{G}|\vec{A}) & \propto  exp \left( -\frac{1}{2} (\vec{G}-\textbf{K}\vec{A})^T diag\{ \frac{1}{\sigma_n^2} \} (\vec{G}-\textbf{K}\vec{A})  \right)
 \\
 & = exp \left( -\frac{1}{2} (\vec{G}-\vec{F})^T diag\{ \frac{1}{\sigma_n^2} \} (\vec{G}-\vec{F})  \right)
 \\
 & = exp(- L(\vec{F},\vec{G}))  
 \end{split}
\end{align}

\subsubsection{The prior distribution}

Since $A(\omega)$ is positive definite $\vec{A}$ has all properties neccesary to apply the form of the MEM prior. 
\begin{equation}\label{eq:prior}
 p(\vec{A}|\alpha,\vec{m}) \propto exp(\alpha S(\vec{A},\vec{m})
\end{equation}

\subsubsection{The posterior distribution}

Putting all things together we found an expression for the probability of $\vec{A}$ to be the true spectral function in 
case observing the data $\vec{G}$ given $\alpha$ and $\vec{m}$.
\begin{equation}\label{eq:posterior}
 p(\vec{A}|\alpha,\vec{m},\vec{G}) \propto  exp(\alpha S - L(\vec{F},\vec{G})) = exp(Q(\vec{A}))
\end{equation}
Therefore for given $\alpha$ and $\vec{m}$ we can calculate the most probable $\vec{\hat{A}}$. However this tells us nothing about
the plausibility of the values for $\alpha$ and $\vec{m}$ which have significant influence of the obtained results.

\subsubsection{Approaches to treat $\alpha$}

We will present 2 common ways how to deal with the uncertainty about how to choose $\alpha$. The so called \textit{Classic} 
and \textit{Bryan's method}. 
\\
If we introduce $p(\alpha$) the prior probability distribution for $\alpha$ (which we assume to be independen on $\vec{m}$ 
we can find a posterior distribution for $\alpha$ using~(\ref{eq:posterior}) as
\begin{equation}
 p(\alpha | \vec{G},\vec{m}) = \int d\vec{A} p(\alpha,\vec{A}|\vec{G},\vec{m}) = \int d\vec{A} p(\vec{A}|\alpha,\vec{m},\vec{G})p(\alpha)
\end{equation}

\noindent Making a gaussian approximation $exp(Q(\vec{A})) \approx exp(Q(\vec{\hat{A}}) + \frac{1}{2} \delta \vec{A}^T \nabla \nabla Q(\vec{\hat{A}}) \delta \vec{A})$
of~(\ref{eq:posterior}) we can approximate the intragral by 

\begin{equation}\label{eq:prob_of_alpha}
 p(\alpha|\vec{G},\vec{m}) \propto  \prod_m  \left( \frac{\alpha}{\alpha + \lambda_m} \right)^{1/2}  exp(Q(\vec{\hat{A}}(\alpha))p(\alpha)
\end{equation}
where $\lambda_m$ are the eigenvalues of $diag\{ \vec{A}^{1/2}\} \nabla \nabla L(\vec{A}) diag\{\vec{A}^{1/2}\} $ 
evaluated at $\vec{\hat{A}}(\alpha)$.
\\
\noindent The \textit{Classic method} is now to choose $\hat{\alpha}$ which maximizes~(\ref{eq:prob_of_alpha}). Seting 
$\partial_{\alpha}p(\alpha|\vec{G},\vec{m}) = 0$ and assuming $\partial_{\alpha}\lambda_m \approx 0$ and that the prior
$p(\alpha)$ will be owerwhelmed by the data, this leads to 
\begin{equation}\label{eq:classic_method_constraint}
 -2\hat{\alpha}S \approx \sum_m \frac{\lambda_m}{\lambda_m +\hat{\alpha}} 
\end{equation}

\noindent \textit{Brayn's method} in contrast tries to use the wohle information contained in $p(\alpha|\vec{G},\vec{m})$. 
Instead of choosing one single value for $\alpha$ one takes the expection value of $\vec{\hat{A}}(\alpha)$.
\begin{equation}\label{eq:brayn_method_constraint}
 \vec{\overline{A}} = \int d\alpha \vec{\hat{A}}(\alpha) p(\alpha|\vec{G},\vec{m})
\end{equation}

